
\doublespacing
\chapter*{Remerciements---Acknowledgements}
\addcontentsline{toc}{chapter}{Remerciements---Acknowledgements}
Cette thèse est l'aboutissement d'un long parcours qui n'aurait pas pris cette forme sans un ensemble de rencontres, de discussions, de soutiens.\\

En premier lieu, je tiens à remercier mon directeur de thèse, David Martimort. Il a été d'un grand soutien durant toute ma thèse, m'a pointé à de nombreuses reprises dans la bonne direction lorsque je bloquais, m'a fait découvrir sa science du calembour, parfois raté, souvent très drôle. Il a aussi été un exemple pour sa rigueur et son exigence envers ce que l'économie devrait selon lui produire. Il a aussi et surtout été d'une patience infinie avec moi. Il faut à mon sens un calme à toute épreuve pour entendre son étudiant annoncer qu'il va lancer sa startup avant d'avoir fini sa thèse. Il en faut encore plus pour rester bienveillant à chaque instant alors que ce même étudiant met au final 6 ans pour soutenir sa thèse. Je ne saurais trop le remercier. \\

Je tiens à remercier Jérôme Pouyet, qui travaillait au même étage que moi, qui m'a donné un grand nombre de références et avec qui nous avons partagé beaucoup de discussions. Mais surtout, je voudrais le remercier pour avoir été celui qui m'a convaincu de faire une thèse après m'avoir enseigné pendant le master APE, de m'avoir recommandé à mon directeur de thèse et d'avoir accepté d'être rapporteur de cette thèse.\\

Je voudrais également remercier les membres de mon comité de thèse, Philippe Choné et Laurent Lamy. Ils ont suivi mon travail durant ces années, m'ont toujours poussé vers plus de clarté dans mon travail, et ont été bien plus compréhensifs que je ne le méritais un certain jour où nous devions nous retrouver pour une réunion sur ma thèse, et où j'ai réussi à arriver avec une heure de retard. Je voudrais également remercier Jean-Christophe Poudou qui a bien voulu être rapporteur dans un délai très court.\\

Je voudrais également rendre hommage à Henri de Belsunce, mon coauteur pour les chapitres empiriques. J'ai une pensée pour les semaines que nous avons passées ensemble à Paris ou à Munich, à vivre essentiellement comme colocataires de court terme, à travailler sans relâche, à s'arracher comme il se doit les cheveux sur des erreurs absurdes dans nos codes, à travailler debout au tableau, allongé devant un vidéoprojecteur dans une salle de conférence à 3 heures du matin, ou encore dans nos salons. \\

Nikolas Wölfing m'a mis sur la piste de l'analyse de données fonctionnelles et m'a invité plusieurs fois à participer à la conférence sur l'économie de l'énergie de Mannheim. Grâce à lui, j'ai pu confronter mes idées à celles de beaucoup d'autres chercheurs spécialisés dans les questions que je traite dans cette thèse. \\

Une thèse ne se résume pas à un ensemble de résultats obtenus devant un écran ou une feuille, mais aussi à une vie entre collègues, et je voudrais ici évoquer mes co-bureaux à PSE ou au CREST: Etienne Chamayou, Julien Combe, Andreea Enache, Perrin Lefebvre, Manuel Marfan, Daria Shakourzadeh. Nous aurons eu un nombre incalculable de discussions de maths, d'économie, de politique, de philosophie, de religion ou encore de physique qui m'auront enrichi académiquement, mais bien plus largement encore. \\

Dans l'environnement très actif de l'École d'Économie de Paris, mais aussi du CREST, j'ai pu rencontrer et discuter avec beaucoup de chercheurs, qui au détour d'une remarque m'ont permis d'aiguiser un argument, de comprendre un concept, ou de suivre une piste bibliographique, merci à Bernard Caillaud, Olivier Compte, Gabrielle Demange, Pierre Fleckinger, Jeanne Hagenbach, Philippe Jehiel, Frederic Koessler, Laurent Linnemer.\\

Je voudrais évoquer Météo France qui m'a fourni les données dont j'avais besoin pour évaluer les impacts de la météo sur le marché de l'électricité, et l'ENPC qui m'a accordé une bourse de thèse. Par ailleurs, faire une thèse en 6 ans induit un certain nombre de détails administratifs à régler, j'aimerais remercier tout particulièrement Sylvie Lambert et Chantal Dekayser, grâce à qui mon inscription pour soutenir ma thèse a abouti. Durant ma thèse, j'ai pu compter sur l'aide des équipes adminitratives de PSE, notamment France Artois-Mbayé, Béatrice Havet ou encore Isabelle Lelièvre.\\

Je voudrais remercier particulièrement Arthur Silve, qui aura été mon colocataire en pointillés pendant deux ans, et qui en tant que docteur en économie quelques années plus tôt que moi, m'a aidé et m'a poussé à finir. Merci pour ces nombreuses discussions autour d'un café le matin, et aussi pour un certain pep talk quelques semaines avant ma pré-soutenance de thèse. Merci à l'autre Arthur, Edouard, et Léonard pour un nombre conséquent d'heures cumulées à parler de ma thèse et de son avancement, quand il n'était pas clair que je finirais cette thèse, et que j'étais allergique à toute discussion sur le sujet.\\

Je n'aurais pas commencé cette thèse, et ne l'aurais pas fini sans mes amis, qui m'ont soutenu, fait rire, fait boire, qui m'ont accompagné pendant ces années, mais aussi avant, en physique ou en économie, je veux les saluer ici : Anasuya, Charles, Charlotte, Claire, Edouard, Gabriel, Grégoire, Guillaume, Jean-Arthur, Léonard, Mathieu, Pierre-Alain, Pu, Sébastien, merci pour tout, je vous dois beaucoup. \\

Pendant la deuxième partie de cette thèse, j'ai consacré la plupart de mon temps à un projet de startup, et je voudrais remercier ici mes associés, Benjamin et Florian, grâce à qui j'ai appris beaucoup, humainement, mais aussi sur l'énergie, l'informatique (il m'est maintenant terriblement douloureux de relire le code informatique produit pour cette thèse, et c'est tant mieux), et qui ont accepté sans broncher les périodes où je les abandonnais pour conclure cette thèse.\\

Je n'aurais assurément pas été à l'ENS, ce qui m'a ensuite emmené vers cette thèse, sans avoir croisé le chemin de professeurs qui m'ont marqué, orienté, influencé. Merci à Mme. Truong, Mme. Courtaud, Mme Casalis, Mr. Reydellet, Mr Douady, Mr Xu, Mr. Joanny, Mr. Cohen, Mr. Piketty, Mr Gilboa.\\

Je ne serais pas qui je suis sans l'environnement familial aimant, bienveillant, et stimulant construit par mes parents, Emmanuelle et Olivier. Grâce vous soit rendu d'avoir subi un nombre de "pourquoi" incalculable, de m'avoir toujours poussé à faire ce qui me plaisait, et d'avoir été là, tout simplement. Je ne serais pas non plus qui je suis sans ma s\oe{}ur, Hortense, qui a subi un frère un peu écrabouilleur pendant notre adolescence, et avec qui nous avons partagé beaucoup de soirées, de vacances et de promenades avec Rune. Je voudrais aussi faire un clin d'\oe{}il aux habituels de la Vendée, ils se reconnaîtront.\\

Je voudrais enfin remercier Morgan, pour ce que nous avons de rare ensemble, nos projets, nos rigolades, nos petits plats et le fait que grâce à elle, je ne sois que le plus nerd, mais pas le plus geek. \\
