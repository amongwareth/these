\chapter*{Summary}
\addcontentsline{toc}{chapter}{Summary}
\cleardoublepage

\section*{Chapter 1: }

The first chapter focuses on what the introduction of ramping costs in a theoretical framework brings to the table. Ramping costs represent the fact that electricity suppliers incur costs when their production varies over time. Our main contribution is to build and justify how these ramping costs can be tackled theoretically. First, we note that going to a continuous time descritption of the problem allows us to bring to the litterature about supply function equilibria powerful mathematical tools mostly used in option pricing, that is stochastic dynamics: we want to model ramping costs, i.e. costs associated to the variation in production, while retaining the key ingredient brought by \cite{KM}, i.e. the uncertainty, through the use of brownians, and more precisely, It\={o} processes. In so doing we face the issue that one cannot derive a brownian, and bring our second contribution, a physical argument about how power plants function that effectively operates as a low pass filter on our stochastic processes, and allow us to continue to build a tractable model of ramping costs under uncertainty. Third, we find in the litterature a specification of It\={o} processes that allows the model to remain tractable. \\

From these technical contributions we obtain our economic contributions in having a rich tractable model that yields results that contrast strongly with past results from the litterature. First, in the specific case of linear demand and linear costs we obtain a unique Nash equilibria, which contrasts with the usual continuum of Nash equilibria in the supply function equilibria litterature. Second, our solutions are not ex-post optimal, meaning that gathering information about the expected future evolution of demand yields different optimal strategies for suppliers, which in turn means that producers in our framework have a motive for submitting different supply functions from one time step to the next. Third, we have closed form solutions which yield specific predictions about the evolution of bids under uncertainty, namely that when uncertainty increase, suppliers submit steeper supply schedules in order to transmit more of these shocks to changes in price and not quantities, which are costly due to the existence of ramping costs. Finally, and less importantly, our framework justifies the existence of negative prices \footnote{Note that such negative prices happen, a few hours a year for example in France or Germany, for example in 2017 there were 146 such hours on 24 days in Germany \cite{epexnegP}} by producers being willing to pay consumers to consume more in order to avoid facing large variations in production, in contrast to everywhere positive schedules in the case of the supply function equilibria litterature.\\

\section*{Chapter 2: }

In the second chapter our main focus is on analyzing our data, on building a way to describe it, and on building proxies for the uncertainty that producers face about the residual demand they have to anticipate when bidding on the day-ahead market. \\

First, we note that aggregate supply functions on the day ahead market cannot be well captured by parametric functions. Therefore we devise a way to describe them non-parametrically: we note that although they cannot be captured parametrically, they still have a rough S shape, and therefore four main parts, two extremal sections, and two interior ones separated by the inflection point of the curve in its middle secion. We define the transition points between these sections as the points of maximal absolute value for the derivative and second derivative of the supply schedules. This definition relies on kernel density estimates, and is therefore non-parametric. We observe that by using 5 such points, we are able to capture about 98\% of the intrisic variability of the supply schedules, and stop there although our method can be used to define more non parametric points. This method allows us to define points that we consider comparable across auctions, that allow use to perform cross-sectional analysis of our data in the third chapter. \\

Second, we build proxies for the amount of weather uncertainty that producers face and variables that capture information that suppliers have before bidding and should therefore be controlled for. For the information available to suppliers, we note that the effect of weather on the demand, and more importantly temperature, is well understood and that we need to control for it. To do so we build an effective temperature for France, as an average of the localised temperature weighted by the population of the spatial region considered, in order to capture the overall effect temperature has on heating.\footnote{France has a high level of electric heating overall, which means that demand for electricity is quite sensitive to temperature.} The rest of our focus is on building a proxy for the uncertainty concerning renewable production. To do so we analyze spatialized wind and sunlight data, and study it's spatial structure. We argue that spatial autocorrelation is a proxy for the uncertainty associated with weather forecasts, noting that if this data displays more spatial gradients, it is likely to be of a lesser quality due to the numerical nature of the weather simulations used to predict the weather, and therefore more uncertain.\\

Our contribution in the second chapter is to provide a non parametric way to define comparable points across auctions, and a measure of the uncertainty associated with weather forecasts.\\

\section*{Chapter 3: }

In this empirical chapter, we study the impact that uncertainty about the demand plays on the shape of the aggregate supply functions bidded by suppliers on the French electricity market. We segment our analysis to different parts of the supply functions in order to show how the overall shape changes with respect to our explanatory variables. We test some of the predictions from our first chapter, mainly that the supply function should see its slope increase when uncertainty increases. \\

We note that the main uncertainty is about the shape of the demand schedules itself. Therefore we consider data available to the producers and regress the demand schedules on these variables. Next, we study the residuals of these regressions, and more specifically note that they are heteroskedastic. We leverage this, regressing the square of these residuals on our variables, in order to predict the expected amplitude of the residuals, that is the amplitude of the uncertainty of the demand schedule regression.\\

We then study the effect of our different proxies for uncertainty on the slope of the supply schedules, and note that if our proxies about the weather uncertainty (through the channel of renewable production) have the expected effect, the results are less clear cut for our residuals on the demand schedules. As we are working with full blown schedules in the quantity-price plane, we perform our residual analysis both on the prices and the quantities. We therefore obtain estimates for the uncertainty pertaining to the position of a given point of our demand schedule either in price or in quantity. In our theoretical framework, we make the strong assumptions that demand schedules are linear, and that demand shocks are additive, i.e. they do not impact the slope of the demand schedules. These assumptions yield that we cannot differentiate between shocks in price or quantity, and that they should have effects in the same direction: more uncertainty implying steeper supply curves to reduce the amount of fluctuations in production. However we observe that the effects of price and quantity uncertainty as estimated by our residuals' method yield opposite effects. Both of these assumptions, although required to obtain closed form results, are clearly not satisfied by our data, and we think that this is a clear path for improvement of the model.  \\

The contribution of the third chapter is to provide a way to estimate the uncertainty about the demand schedules faced by suppliers, and to estimate how this uncertainty affects the shape of the supply schedules at different points along its overall length, i.e. we provide a framework to describe how the functional form of schedules is affected by estimates of the uncertainty faced by suppliers.\\