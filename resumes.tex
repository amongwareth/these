\chapter*{Summary}
\addcontentsline{toc}{chapter}{Summary}
\cleardoublepage

\section*{Chapter 1: }

In this theoretical chapter we build on existing supply function equilibria (SFE) models under uncertainty by taking into account ramping costs. Ramping costs represent the fact that electricity suppliers incur costs when their production varies over time. By introducing a way to take into account these costs yet still retain the uncertainty that is key to the SFE framework, we are able to show that in the specific case of linear demand and linear costs we obtain a unique Nash equilibria. This comes in stark contrast to the usual continuum of equilibria. We further solve the model in the case of a symmetric Nash equilibria with suppliers expecting changes in the demand shocks over time, which in turn drives the dynamics of their bidds.

\section*{Chapter 2: }

In this methodological chapter, we first introduce results from the functional data analysis literature in order to provide a non parametric method to perform cross-sectional analysis of the french electricity market data and secondly, we introduce methods to leverage localized weather data to estimate the relative induced demand uncertainty at the national level in order to further study the impact that uncertainty plays in the bidding strategies of electricity producers.

\section*{Chapter 3: }
In this empirical paper, we study the impact that uncertainty about the demand plays on the shape of the aggregate supply functions bidded by suppliers on the French electricity market. We segment our analysis to different parts of the supply functions in order to show how the overall shape changes with respect to our explanatory variables. We test some of the predictions from our first chapter, mainly that the supply function should see its slope increase when uncertainty increases. We obtain contrasting results : most of our direct predictions are of the correct sign but not always significant, and some regressors that we don't have direct predictions for have signs that are the opposite of the ones we deem to be coherent with our predictions. 