\renewcommand{\thesection}{\arabic{chapter}.\arabic{section}}

\chapter*{General conclusion}
\addcontentsline{toc}{chapter}{General conclusion}
\label{chap:ccl}
\cleardoublepage
\doublespacing

This thesis focused on the effect of ramping costs on the electricity market at a theoretical level, and then on the empirical analysis of market data to test the theoretical predictions.\\

The first chapter focuses on what the introduction of ramping costs in a theoretical framework brings to the table. Ramping costs represent the fact that electricity suppliers incur costs when their production varies over time. Our main contribution is to build and justify how these ramping costs can be tackled theoretically. First, we note that going to a continuous time descritption of the problem allows us to bring to the litterature about supply function equilibria powerful mathematical tools mostly used in option pricing, that is stochastic dynamics: we want to model ramping costs, i.e. costs associated to the variation in production, while retaining the key ingredient brought by \cite{KM}, i.e. the uncertainty, through the use of brownians, and more precisely, It\={o} processes. In so doing we face the issue that one cannot derive a brownian, and bring our second contribution, a physical argument about how power plants function that effectively operates as a low pass filter on our stochastic processes, and allow us to continue to build a tractable model of ramping costs under uncertainty. Third, we find in the litterature a specification of It\={o} processes that allows the model to remain tractable. \\

From these technical contributions we obtain our economic contributions in having a rich tractable model that yields results that contrast strongly with past results from the litterature. First, in the specific case of linear demand and linear costs we obtain a unique Nash equilibria, which contrasts with the usual continuum of Nash equilibria in the supply function equilibria litterature. Second, our solutions are not ex-post optimal, meaning that gathering information about the expected future evolution of demand yields different optimal strategies for suppliers, which in turn means that producers in our framework have a motive for submitting different supply functions from one time step to the next. Third, we have closed form solutions which yield specific predictions about the evolution of bids under uncertainty, namely that when uncertainty increase, suppliers submit steeper supply schedules in order to transmit more of these shocks to changes in price and not quantities, which are costly due to the existence of ramping costs. Finally, and less importantly, our framework justifies the existence of negative prices \footnote{Note that such negative prices happen, a few hours a year for example in France or Germany, for example in 2017 there were 146 such hours on 24 days in Germany \cite{epexnegP}} by producers being willing to pay consumers to consume more in order to avoid facing large variations in production, in contrast to everywhere positive schedules in the case of the supply function equilibria litterature. These results open the door to models being able to differentiate between day-ahead and intraday markets and therefore to offer a framework in which their interactions might be possible.\\

In the second chapter our main focus is on analyzing our data, on building a way to describe it, and on building proxies for the uncertainty that producers face about the residual demand they have to anticipate when bidding on the day-ahead market. \\

First, we note that aggregate supply functions on the day ahead market cannot be well captured by parametric functions. Therefore we devise a way to describe them non-parametrically: we note that although they cannot be captured parametrically, they still have a rough S shape, and therefore four main parts, two extremal sections, and two interior ones separated by the inflection point of the curve in its middle secion. We define the transition points between these sections as the points of maximal absolute value for the derivative and second derivative of the supply schedules. This definition relies on kernel density estimates, and is therefore non-parametric. We observe that by using 5 such points, we are able to capture about 98\% of the intrisic variability of the supply schedules, and stop there although our method can be used to define more non parametric points. This method allows us to define points that we consider comparable across auctions, that allow use to perform cross-sectional analysis of our data in the third chapter. \\

Second, we build proxies for the amount of weather uncertainty that producers face and variables that capture information that suppliers have before bidding and should therefore be controlled for. For the information available to suppliers, we note that the effect of weather on the demand, and more importantly temperature, is well understood and that we need to control for it. To do so we build an effective temperature for France, as an average of the localised temperature weighted by the population of the spatial region considered, in order to capture the overall effect temperature has on heating.\footnote{France has a high level of electric heating overall, which means that demand for electricity is quite sensitive to temperature.} The rest of our focus is on building a proxy for the uncertainty concerning renewable production. To do so we analyze spatialized wind and sunlight data, and study it's spatial structure. We argue that spatial autocorrelation is a proxy for the uncertainty associated with weather forecasts, noting that if this data displays more spatial gradients, it is likely to be of a lesser quality due to the numerical nature of the weather simulations used to predict the weather, and therefore more uncertain.\\

Our contribution in the second chapter is to provide a non parametric way to define comparable points across auctions, and a measure of the uncertainty associated with weather forecasts.\\

In this empirical chapter, we study the impact that uncertainty about the demand plays on the shape of the aggregate supply functions bidded by suppliers on the French electricity market. We segment our analysis to different parts of the supply functions in order to show how the overall shape changes with respect to our explanatory variables. We test some of the predictions from our first chapter, mainly that the supply function should see its slope increase when uncertainty increases. \\

We note that the main uncertainty is about the shape of the demand schedules itself. Therefore we consider data available to the producers and regress the demand schedules on these variables. Next, we study the residuals of these regressions, and more specifically note that they are heteroskedastic. We leverage this, regressing the square of these residuals on our variables, in order to predict the expected amplitude of the residuals, that is the amplitude of the uncertainty of the demand schedule regression.\\

We then study the effect of our different proxies for uncertainty on the slope of the supply schedules, and note that if our proxies about the weather uncertainty (through the channel of renewable production) have the expected effect, the results are less clear cut for our residuals on the demand schedules. As we are working with full blown schedules in the quantity-price plane, we perform our residual analysis both on the prices and the quantities. We therefore obtain estimates for the uncertainty pertaining to the position of a given point of our demand schedule either in price or in quantity. In our theoretical framework, we make the strong assumptions that demand schedules are linear, and that demand shocks are additive, i.e. they do not impact the slope of the demand schedules. These assumptions yield that we cannot differentiate between shocks in price or quantity, and that they should have effects in the same direction: more uncertainty implying steeper supply curves to reduce the amount of fluctuations in production. However we observe that the effects of price and quantity uncertainty as estimated by our residuals' method yield opposite effects. Both of these assumptions, although required to obtain closed form results, are clearly not satisfied by our data, and we think that this is a clear path for improvement of the model.  \\

The contribution of the third chapter is to provide a way to estimate the uncertainty about the demand schedules faced by suppliers, and to estimate how this uncertainty affects the shape of the supply schedules at different points along its overall length, i.e. we provide a framework to describe how the functional form of schedules is affected by estimates of the uncertainty faced by suppliers.\\

\section*{Avenues of research}

The work presented in this thesis opens new possible avenues of research, that we will outline here. 

\subsection*{Theoretical model}

\begin{itemize}
\item Generalize the functional forms of the demand: we developed our model in the context of linear demand functions, and finding either general results for, say positive and decreasing demand functions, would lend more support to our results. It would already be interesting to find whether these results hold for other specific functional forms for the demand functions. The issue is that the second order differential equations do not belong to solved for classes of equations in the cases that were tested in the course of this thesis (power demand functions for example). It is therefore unlikely that analytical results can be obtained, however numerical approaches could prove useful in this context. 

\item Study the impact of other stochastic processes: our results hold in the case of stochastic shocks leading to an equilibrium distribution of a quadratic form. The processes that we use to obtain our results are part of a larger class of processes, which can exhibit richer caracteristics, for example assymetric distributions. As previously, the analytical nature of our results rely partly on the specific choice of stochastic process we made, therefore analytical results are unlikely, but numerical approaches could shed light on the effect of skewed distributions.

\item Study how a time discrete approache converges towards our continuous time one: the derivation is doable in the case of discrete states for demand shock. A toy model not presented in this thesis was derived in the case of a two period two valued shocks model. Although the results are consistent with those of our continuous model, the expressions derived analytically are already horribly tedious. It is therefore once again a strand of analysis that could profit from a numerical approach.

\item The interaction between intraday and day-ahead markets: knowing that when one bids on the day-ahead market, it will still be possible to adjust one's position tomorrow on the intraday market is bound to impact the strategies of the suppliers. Trying to tackle this problem, if challenging, could prove very interesting. Here are key ingredients that should be taken into account: 

First, intraday bids can be submitted anytime during the day with an expiry date attached. Therefore, there is tension between, on the one hand, the will to start and correct the outcome of the day-ahead market as soon as new information enters about the demand shocks, so as to increase the likelihood for another agent to buy the intraday bid, and on the other hand, the will to wait and see as information enters to be as precise as possible on the submitted bid to correct the outcome of the day-ahead market, but therefore decreasing the likelihood to find a buyer. 

Second, the ramping costs associated with changing production are incurred only after the net of the day-ahead market and intraday market is fixed. 

\item Implement the actual market clearing algorithm and study it numerically.
\end{itemize}

\subsection*{Empirical analysis}

\begin{itemize}
\item Take into account the block orders: these orders impact the bids and should be accounted for.

\item Study in more detail the overall function without restriction to only 5 points, which should be doable with the increase in computational power

\item Leverage the difference between weather prediction data and observations for more accurate weather uncertainty.

\item Take into account the uncertainty associated with international interconnexions. 

\item Analyze the individual submitted points on the aggregate supply schedules to find whether it is possible to attach some of them to specific power plants with any certainty.
\end{itemize}



 